\documentclass[12pt]{article}
\usepackage{trishan2}

\title{\textbf{Assignment-5}}
\author{Trishan Mondal, Soumya Dasgupta, Aaratrick Basu}
\date{}

\begin{document}
\maketitle

\section*{Problem \textcolor{maroon}{5.14}}
We will begin with the assumption, the underlying field $k$ is infinte and algebraically closed (according to contexts). The property of lines passing through points is a projective property. So we can take a suitable projective transformation so that $P_1 = [0:0:1]$. Thus, any line passing through this looks like $ax +by = 0$ where $a,b \in k$. The set of lines passing through $P_1$ is $$A = \qty{x+my : m \in k} \cup \qty{y=0}$$ Since, the field is infinite, there is infinitely many elements in $A$. Given two points in $\mathbb{P}^2$ there is a unique line passing through $P_1$ and that point. Thus the set of lines $$L=\qty{\ell \text{ pass through } P_1 \text{ and }P_{i} : 2 \leq i \leq n} \subset A$$ is finite. So there are only finitely many line in the above set. But in $A$ there are infinitely elements. So, there are infinitely many elements in $A \setminus L$.

\vspace*{0.2cm}

\noindent Since $P_1$ is a simple point of $F$, there is a tangent $T$ at $P$ so that the tangent $T$ don't contained in $V(F)$ (or $F$). From the problem \textcolor{maroon}{5.12} we can say, $$\sum I(P;F\cap T)=n$$ where $n = \deg F$. Thus, If we take $P_2,\cdots,P_m$ be the other intersection points (here $m \leq n$) of $T$ and $F$, by the previous calculation we can say there exists infinitely many lines through $P$ don't intersect $F$ at $P_i$ ($i >1$). These lines are transversal to $F$. $\hfill \blacksquare$

\section*{Problem \textcolor{maroon}{5.18}}

Let us consider the general equation of conic in $\mathbb{P}^2$, that is $$Ax^2+By^2+Cz^2+Exy+Fyz + G zx =0$$ Since the pont $[0:0:1]$ and $[0:1:0]$, $[1:0:0]$ passes through  the above conic we can say, $A=B=C=0$. Thus the equation of conic reduces to $E xy + Fyz+Gzx=0$. Also the points $[1:1:1]$ and $[1:2:3]$ passes through the curve. So we have the following linear equations, 
\begin{align*}
     E+F+G &=0\\
     2E+6F +3G &=0 \\
     \implies \begin{pmatrix}
        1 & 1 & 1\\
        2 & 6 & 3
     \end{pmatrix} \begin{pmatrix}
        E \\ F \\ G
     \end{pmatrix} & = 0
\end{align*}
Note that the rows of the aboe matrix are linearly independent. So the null space of it must have dimension $1$. Note that $(3, \, -4, \,1)^T$ is a solution to the above matrix equation. Since the dimension of null space is $1$ we can say any other solution must be a scaler multiple of $(3,\, -4, \,1)^T$. So the equation of conic passing through the five  points is $\lambda (3xy -4yz+zx)=0$. This will represent a unique conic in $\mathbb{P}^2$. By contruction the conic is unique! $\hfill \blacksquare$ 

\section*{Problem \textcolor{maroon}{5.25}}

Since the polynomial $F=F_1F_2$ have $c\geq 1$ simple component, the polynomial may not be irreducible. Let, $F=F_1F_2$ and at every point $P$, $m_P(F)= m_P(F_1)+m_P(F_2)$. Thus, \begin{align*}
   \sum_P \frac{m_P(F)(m_P(F)-1)}{2} &= \sum_{P} \frac{(m_P(F_1)+m_P(F_2))(m_P(F_1)+m_P(F_2)-1)}{2} \\
   &= \sum_{P} \frac{m_P(F_1)(m_P(F_1)-1)}{2} + \sum_{P} \frac{m_P(F_2)(m_P(F_2)-1)}{2}  \\&+ \sum_{P}m_P(F_1)m_P(F_2)
\end{align*}
Let, $p = \deg F_1$ and $q = \deg F_2$. If $F_1$ and $F_2$ were irreducible then we must have \begin{align*}
   \sum_P \frac{m_P(F)(m_P(F)-1)}{2} 
   &= \sum_{P} \frac{m_P(F_1)(m_P(F_1)-1)}{2} + \sum_{P} \frac{m_P(F_2)(m_P(F_2)-1)}{2}  \\&+ \sum_{P}m_P(F_1)m_P(F_2) \\
   &\overset{\textcolor{maroon}{\ast}}{\leq} \frac{(p-1)(p-2)}{2}+\frac{(q-1)(q-2)}{2} + pq \\
   &= \frac{(p+q-1)(p+q-2)}{2} +1 \\
   &= \frac{(n-1)(n-2)}{2}+1
\end{align*}
here, $\textcolor{maroon}{\ast}$ comes from the \textcolor{maroon}{corollary 1} of B\'ezout's theorem and theorem of section \textcolor{maroon}{5.4}. In this case we had $c=2$. Now we will proceed using induction. Assume the result is true for some curve with $c-1$ simple components. Again assume $F=F1F_2$ with the degrees mentioned above and $F_1$ has $c-1$-simple components and $F_2$ is irreducible. Thus using induction we have, 
\begin{align*}
   \sum_P \frac{m_P(F)(m_P(F)-1)}{2} 
   &= \sum_{P} \frac{m_P(F_1)(m_P(F_1)-1)}{2} + \sum_{P} \frac{m_P(F_2)(m_P(F_2)-1)}{2}  \\&+ \sum_{P}m_P(F_1)m_P(F_2)\\
   & \leq \frac{(p-1)(p-2)}{2}+c-2+\frac{(q-1)(q-2)}{2} + pq\\
   &= \frac{(p+q-1)(p+q-2)}{2} +c-1 = \frac{(n-1)(n-2)}{2}+c-1
\end{align*}
Thus our induction step is complete. It's not hard to note that a polynomial of degree $n$ can have at most $n$ linear factor, i.e atmost $n$ simple components. Thus $c \leq n$ and hence the final term in the above calculation is bounded above by $n(n-1)/2$. $\hfill\blacksquare$
\end{document} 