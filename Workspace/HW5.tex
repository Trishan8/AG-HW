\documentclass[12pt]{article}
\usepackage{trishan2}

\title{\textbf{Assignment-5}}
\author{Trishan Mondal, Soumya Dasgupta, Aaratrick Basu}
\date{}

\begin{document}
\maketitle

\section*{Problem \textcolor{maroon}{5.5}}
Let $F(X,Y,Z) = \sum_{i=m}^n F_i(X,Z) Y^{n-i}$. Then for $P = [0:1:0]$
\begin{align*}
   m_P(F) &= m_{\varphi(P)}(F_*) \\ 
   &= m_{(0,0)}(\sum_{i=m}^n F_i(X,Z)) \\ 
   &= m.
\end{align*}

A line $L$ is tangent to $F$ if and only if $I(P, F \cap L) > m_p(F)$, thus we must have 
\begin{align*}
   I(P,F \cap L) &= \dim_{k} \mathscr{O}_P(\mathbb{P}^2)/(F_*,L_*) \\
   &= \dim_k \mathscr{O}_P(\mathbb{P}^2)/(F(X/Y,1,Z/Y), L/Y) \\ 
   &= \dim_k \mathscr{O}_{(0,0)}(\mathbb{A}^2)/(F(X,1,Z), L(X,1,Z)) \mathscr{O}_{(0,0)}(\mathbb{A}^2) \\ 
   &= I((0,0), F(X,1,Z) \cap L(X,1,Z)).
\end{align*}
Thus we get that $I((0,0), F(X,1,Z) \cap L(X,1,Z)) > m$, hence $L(X,1,Z)$ is tangent to $F(X,1,Z)$, thus it must be a factor of $F_m(X,Z)$ (by defnition of tangent for an affine curve). Therefore, the tangents to $F$ are determined by the factors of $F_m(X,Z)$.

\section*{Problem \textcolor{maroon}{5.7}}
Let $F$ and $G$ be two plane curves with no common components. Let $L$ be a line not contained in $V(FG) \subseteq \mathbb{P}^2$. Then by problem 12, we know that $F \cap L$ and $G \cap L$ are finite. Now there exists a projective transformation that takes the line $L$ to $Z$. Then under this projective transformation we know that intersection numbers of $F$ and $G$ are preserved. And we have 
\begin{align*}
   F \cap G = (\underbrace{(F \cap U) \cap (G \cap U)}_{A}) \cup (\underbrace{(F \cap Z) \cup (G \cap Z)}_{B})
\end{align*}
where $U = \{ [x:y:z] \in \mathbb{P}^2 \mid z = 1 \}$. Note that $B$ is finite by the choice of the line $L$. Now $F \cap U$ and $G \cap U$ are affine curves given by $f = F(X,Y,1)$ and $g = G(X,Y,1)$. Now since $F$ and $G$ does not have any common component so does $f$ and $g$ (since otherwise we would have $hp = f$ and $hq = g$ for some $h,p,q \in k[X,Y]$, then $h^* p^* = F$ and $h^* q^* = G$, but then $h^*$ is a common component of $F$ and $G$, contradiction!). But we have previously shown that if two affine curves have no common component then $f \cap g$ is finite. Hence both $A$ and $B$ are finite, thus $F \cap G$ is finite. 

\section*{Problem \textcolor{maroon}{5.12}}
\textbf{Part (a).} Let $P \in [0:1:0] \in F$ where $F$ is a curve of degree of $n$. Let $F(X,Y,Z) = \sum_{i=0}^n F_i(Y,Z)X^i$ with $F_i$ is a form of degree $n-i$ with $F_0 \neq 0$ and let $F_0(Y,Z) = \sum_{i=m}^{m+k} a_i Y^i Z^{n-i}$ (with $m,k \geq 0$ and $m+k \leq n-1$, there is no $Y^n$ term as $P = [0:1:0] \in F$). 
\begin{align*}
   \sum_{P \in \mathbb{P}^2} I(P, F \cap X) &= \sum_{P \in F_0 \cap X} I(P, F_0 \cap X) \\ 
   &= \sum_{P \in F_0 \cap X \cap U_1} I(P, F_0 \cap X) + I([0:0:1], F_0 \cap X) \\ 
   &= \sum_{t \in k} I([0:1:t], F_0 \cap X) + I([0:0:1], F_0 \cap X) \\
   &= \sum_{t \in k} \dim_k \left( \mathscr{O}_{[0:1:t]}(\mathbb{P}^2)/(F_{0*} \cap X_*)\right) + \dim_k \left( \mathscr{O}_{[0:0:1]}(\mathbb{P}^2)/(F_{0*} \cap X_*) \right) \\ 
   &= \sum_{t \in k} \dim_k\left( \mathscr{O}_{(0,t)}(\mathbb{A}^2)/(F_0(1,Z),X)\mathscr{O}_{(0,t)}(\mathbb{A}^2) \right) + \dim_k \left( \mathscr{O}_{(0,0)}(\mathbb{P}^2)/(F_0(Y,1),X)\mathscr{O}_{(0,0)}(\mathbb{A}^2) \right) \\ 
   &= \sum_{t \in k} I((0,t), F_0(1,Z) \cap X) + \mathrm{ord}_{(0,0)}^X(F_0(Y,1)) \\ 
   &= \sum_{P \in F_0(1,Z) \cap X} I(P, F_0(1,Z) \cap X) + \mathrm{ord}^X_{(0,0)}(F_0(Y,1)) \\ 
   &= \deg F_0(1,Z) \deg X + m \\ 
   &= (n-m) + m = n.
\end{align*}
Hence we have proved that $\sum_{P \in \mathbb{P}^2} I(P, F \cap X) = n$. 

\ 

\noindent\textbf{Part (b).} Now if $L$ is not a line contained in $F$, we can find a projective transformation taking $P \in F \mapsto [0:1:0]$ and $L \mapsto X$, then by part (a), we get that 
\begin{align*}
   \sum_{P \in \mathbb{P}^2} I(P, F \cap L) = n.
\end{align*}

\section*{Problem \textcolor{maroon}{5.14}}
We will begin with the assumption, the underlying field $k$ is infinte and algebraically closed (according to contexts). The property of lines passing through points is a projective property. So we can take a suitable projective transformation so that $P_1 = [0:0:1]$. Thus, any line passing through this looks like $ax +by = 0$ where $a,b \in k$. The set of lines passing through $P_1$ is $$A = \qty{x+my : m \in k} \cup \qty{y=0}$$ Since, the field is infinite, there is infinitely many elements in $A$. Given two points in $\mathbb{P}^2$ there is a unique line passing through $P_1$ and that point. Thus the set of lines $$L=\qty{\ell \text{ pass through } P_1 \text{ and }P_{i} : 2 \leq i \leq n} \subset A$$ is finite. So there are only finitely many line in the above set. But in $A$ there are infinitely elements. So, there are infinitely many elements in $A \setminus L$.

\vspace*{0.2cm}

\noindent Since $P_1$ is a simple point of $F$, there is a tangent $T$ at $P$ so that the tangent $T$ don't contained in $V(F)$ (or $F$). From the problem \textcolor{maroon}{5.12} we can say, $$\sum I(P;F\cap T)=n$$ where $n = \deg F$. Thus, If we take $P_2,\cdots,P_m$ be the other intersection points (here $m \leq n$) of $T$ and $F$, by the previous calculation we can say there exists infinitely many lines through $P$ don't intersect $F$ at $P_i$ ($i >1$). These lines are transversal to $F$. $\hfill \blacksquare$

\section*{Problem \textcolor{maroon}{5.18}}

Let us consider the general equation of conic in $\mathbb{P}^2$, that is $$Ax^2+By^2+Cz^2+Exy+Fyz + G zx =0$$ Since the pont $[0:0:1]$ and $[0:1:0]$, $[1:0:0]$ passes through  the above conic we can say, $A=B=C=0$. Thus the equation of conic reduces to $E xy + Fyz+Gzx=0$. Also the points $[1:1:1]$ and $[1:2:3]$ passes through the curve. So we have the following linear equations, 
\begin{align*}
     E+F+G &=0\\
     2E+6F +3G &=0 \\
     \implies \begin{pmatrix}
        1 & 1 & 1\\
        2 & 6 & 3
     \end{pmatrix} \begin{pmatrix}
        E \\ F \\ G
     \end{pmatrix} & = 0
\end{align*}
Note that the rows of the aboe matrix are linearly independent. So the null space of it must have dimension $1$. Note that $(3, \, -4, \,1)^T$ is a solution to the above matrix equation. Since the dimension of null space is $1$ we can say any other solution must be a scaler multiple of $(3,\, -4, \,1)^T$. So the equation of conic passing through the five  points is $\lambda (3xy -4yz+zx)=0$. This will represent a unique conic in $\mathbb{P}^2$. By contruction the conic is unique! $\hfill \blacksquare$ 

\section*{Problem \textcolor{maroon}{5.25}}

Since the polynomial $F=F_1F_2$ have $c\geq 1$ simple component, the polynomial may not be irreducible. Let, $F=F_1F_2$ and at every point $P$, $m_P(F)= m_P(F_1)+m_P(F_2)$. Thus, \begin{align*}
   \sum_P \frac{m_P(F)(m_P(F)-1)}{2} &= \sum_{P} \frac{(m_P(F_1)+m_P(F_2))(m_P(F_1)+m_P(F_2)-1)}{2} \\
   &= \sum_{P} \frac{m_P(F_1)(m_P(F_1)-1)}{2} + \sum_{P} \frac{m_P(F_2)(m_P(F_2)-1)}{2}  \\&+ \sum_{P}m_P(F_1)m_P(F_2)
\end{align*}
Let, $p = \deg F_1$ and $q = \deg F_2$. If $F_1$ and $F_2$ were irreducible then we must have \begin{align*}
   \sum_P \frac{m_P(F)(m_P(F)-1)}{2} 
   &= \sum_{P} \frac{m_P(F_1)(m_P(F_1)-1)}{2} + \sum_{P} \frac{m_P(F_2)(m_P(F_2)-1)}{2}  \\&+ \sum_{P}m_P(F_1)m_P(F_2) \\
   &\overset{\textcolor{maroon}{\ast}}{\leq} \frac{(p-1)(p-2)}{2}+\frac{(q-1)(q-2)}{2} + pq \\
   &= \frac{(p+q-1)(p+q-2)}{2} +1 \\
   &= \frac{(n-1)(n-2)}{2}+1
\end{align*}
here, $\textcolor{maroon}{\ast}$ comes from the \textcolor{maroon}{corollary 1} of B\'ezout's theorem and theorem of section \textcolor{maroon}{5.4}. In this case we had $c=2$. Now we will proceed using induction. Assume the result is true for some curve with $c-1$ simple components. Again assume $F=F1F_2$ with the degrees mentioned above and $F_1$ has $c-1$-simple components and $F_2$ is irreducible. Thus using induction we have, 
\begin{align*}
   \sum_P \frac{m_P(F)(m_P(F)-1)}{2} 
   &= \sum_{P} \frac{m_P(F_1)(m_P(F_1)-1)}{2} + \sum_{P} \frac{m_P(F_2)(m_P(F_2)-1)}{2}  \\&+ \sum_{P}m_P(F_1)m_P(F_2)\\
   & \leq \underbrace{\frac{(p-1)(p-2)}{2}+c-2}_{\text{induction step}}+\frac{(q-1)(q-2)}{2} + pq\\
   &= \frac{(p+q-1)(p+q-2)}{2} +c-1 = \frac{(n-1)(n-2)}{2}+c-1
\end{align*}
Thus our induction step is complete. It's not hard to note that a polynomial of degree $n$ can have at most $n$ linear factor, i.e atmost $n$ simple components. Thus $c \leq n$ and hence the final term in the above calculation is bounded above by $n(n-1)/2$. $\hfill\blacksquare$
\end{document} 