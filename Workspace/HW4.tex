\documentclass[12pt]{article}
\usepackage{trishan2}

\title{\textbf{Assignment-4}}
\author{Trishan Mondal, Soumya Dasgupta, Aaratrick Basu}
\date{}

\begin{document}
\maketitle

\section{Problem \textcolor{maroon}{3.6}} 
\begin{Lem}{1} The $F$ and $G$ be forms of degree $r$ and $r+1$ respectively with no common factors in $k[X_1,\dots,X_n]$, then $F + G$ is irreducible.
\end{Lem}

\textit{Proof (of Lemma).} Suppose $F + G$ is reducible then there exists nonconstant polynomials $P,Q \in k[X_1,\dots,X_n]$ such that $F + G = PQ$. Now we consider the homogeneous both of these to get 
\begin{align*}
    X_{n+1} F + G = (F + G)^* = (PQ)^* = P^* Q^*.
\end{align*}
But note that $X_{n+1} F + G \in k(X_1,\dots,X_n)[X_{n+1}]$ is irreducible, hence one of $P^*$ or $Q^*$ is in $k[X_1,\dots,X_n]$. Then by comparing degrees we can WLOG assume that $Q^* \in k[X_1,\dots,X_n]$, and let $P = X_{n+1} R + S$, where $R , S \in k[X_1,\dots,X_n]$. Then we get that 
\begin{align*}
    X_{n+1} F + G = X_{n+1} R Q^* + S Q^* \Rightarrow F = R Q^* \mbox{ and } G = S Q^*
\end{align*}
But this contradicts the fact that $F$ and $G$ have no common factor, hence we get that $F + G$ is irreducible. 

Now coming back to the main problem, suppose we are given tangent lines $L_i$ with multiplicities $r_i$, and we want to find an irreducible curve $F$ such that $L_i$ is a tangent to $F$ with multiplicity $r_i$. Note that $\prod_i L_i^{r_i}$ is a forms of degree $m = \sum_i r_i$. Then we can find a homogeneous polynomial $F_{m+1}$ of degree $m+1$ such that $F_{m+1}$ is not divisible by any of the $L_i$ (such a polynomial obvious exists). But then by the previous lemma $\prod_i L_i^{r_i} + F_{m+1}$ is irreducible, and clearly $F = \prod_i L_i^{r_i} + F_{m+1}$ satisfies the necessary conditions. 

\section{Problem \textcolor{maroon}{3.8}} %Problem 3

\ts{Part (a).} We will first prove it for the case when $P = Q =(0,0)$. In this case the polynomial map $T : \A^2 \to \A^2$ must look like $(f,g)$ where $f$ and $g$ are polynomials vanishing at $(0,0)$. In this case we can wite $f = f_i + \cdots + f_t$, where $f_i  \in k[x,y],i \geq 1$ is homogeneous polynomial of degree $i$. Similarly, we can write for $g$ (as both of them are vanishing at $(0,0)$). If $m = m_P(F)$ then $F = F_m +F_{m+1}\cdots$ again $F_i$ are homogeneous polynomial of degree $m$. Now $F^T = F(f,g)$'s lowest degree will come from $F_m(f,g)$ since both $f,g$ has at-least one degree term we can say, $m_Q(F^T)\geq m_p(F)$.  

\vspace*{0.2cm}

 Now we will use the fact proved in page (33) to prove it for any $P,Q$. Let, $Q\neq (0,0)$ or $Q = T(P)\neq (0,0)$. Let $T_1 : \A^2 \to \A^2$ be the affine transformation that maps $(0,0)$ to $Q$ and $T_2$ be the affine map sends $P$ to $(0,0)$. Note that $T_1  \circ T \circ T_2$ is a polynomial map and it maps $(0,0)$. So by the above calculation we can say, \begin{align*}
    m_P(F) &\leq m_{T_1  \circ T \circ T_2(0,0)}(F^{T_1  \circ T \circ T_2}) \\
    &= m_{T_1  \circ T (Q)}(F^{T_1  \circ T }) \text{ (By result of page 33})\\
    &= m_{T (Q)}(F^{T }) \text{ (By result of page 33})
 \end{align*}

 \vspace*{0.2cm}

 \noindent \ts{Part (b).} Again we will prove it for $P=Q=(0,0)$. Let $T = (f,g)$ and 
 $$
 J_Q T = \begin{pmatrix} \dfrac{\partial f}{\partial X}(Q) & \dfrac{\partial f}{\partial Y}(Q) \\ \dfrac{\partial g}{\partial X}(Q) & \dfrac{\partial g}{\partial Y}(Q) \end{pmatrix}.
 $$
 Assume $J_Q T$ is invertible. Since $J_Q T$ is invertible, we can't have both $\dfrac{\partial f}{\partial X}(Q) = 0$ and $\dfrac{\partial f}{\partial Y}(Q) = 0$ or both $\dfrac{\partial g}{\partial X}(Q) = 0$ and $\dfrac{\partial g}{\partial Y}(Q) = 0$. Again by similar computation of part $(a)$ we have, since $Q = (0,0)$, this implies that the decomposition of $f$ and $g$ into homogeneous polynomials are $f = f_1 + \cdots + f_m$ 
and $g = g_1 + \cdots + g_n$. Thus,
 $$
 F^T = F(f,g) = F_m(f,g) + F_{m+1}(f,g) + \ldots
 $$
 Since the lowest degree forms of $f$ and $g$ are of degree $1$, we have that $T$ does not decrease the degree of the form $F_m(f,g)$. Similarly, $T$ does not decrease the degree of $F_{m+1}(f,g), \cdots$. Therefore we have that $m_{(0,0)}(F^T) = m_{(0,0)}(F)$. Now assume that either $Q = (a_1,b_1) \neq (0,0)$ or $P = (a_2,b_2) \neq (0,0)$. Assume that $J_Q T$ is invertible. Let $T_1$  be the translation that takes $(0,0)$ to $Q$ and $T_2$ be the translation that takes $P$ to $(0,0)$. Then we have
 $$
  d(T_1\circ T\circ T_2) = d(T_1)\circ d(T) \circ d(T_2)
 $$
is also invertible. By the previous case $m_P(F) = m_{(0,0)}(F^{T_1\circ T\circ T_2})$and the similar computation of multiplicities we can say $m_P(F^T)=m_Q(F)$. And hence our proof is complete. 

\vspace*{0.2cm}

\noindent \ts{Part (c).} If $F= Y-X^2$ and $T =(X^2,Y)$, $P=Q =(0,0)$ we can see $m_P(F^T)=m_P(Y-X^4)=m_Q(F)=m_Q(Y-X^2)$. But the jacobian of $T$ is not invertible at $(0,0)$,as it is given by the matrix$\begin{pmatrix}
    0 & 0\\
    0 & 1
\end{pmatrix}$.\Qed

\section{Problem \textcolor{maroon}{3.13}} 

WLOG assume $P = (0,0)$, then we know that 
\begin{align*}
    \dim_k \left( \mathscr{O}/\mathfrak{m}^n \right) = \dim_k \left( k[X,Y]/(F,I^n)\right)
\end{align*}
where $I = (X,Y) \subseteq k[X,Y]$. Now as multiplicity of $F$ is $m_p(F)$, we have $F \in I^{m_P(F)}$ and hence we get that for $n \leq m_p(F)$, $F \in I^n$, thus $(F,I^n) = I^n$. But then we get that 
\begin{align*}
    \dim_k \left( k[X,Y]/(F,I^n)\right) = \dim_k\left(k[X,Y]/I^n\right) = \binom{n+1}{2} 
\end{align*}
Now from the exact sequence 
\begin{align*}
    0 \to \mathfrak{m}^n/\mathfrak{m}^{n+1} \to \mathscr{O}/\mathfrak{m}^{n+1} \to \mathscr{O}/\mathfrak{m}^n \to 0
\end{align*}
we get that for $n \leq m_p(F)$.
\begin{align*}
    \dim_k \left( \mathfrak{m}^n/\mathfrak{m}^{n+1} \right) &= \dim_k \left( \mathscr{O}/\mathfrak{m}^{n+1}\right) - \dim_k(\mathscr{O}/\mathfrak{m}^n) \\ &= \binom{n+2}{2} - \binom{n+1}{2} \\ &= n+1
\end{align*}

In the proof of \textbf{Theorem 2, page 35 (Algebraic Curves, Fulton)}, we have already seen that 
\begin{align*}
    \dim_k \left(k[X,Y]/(F,I^n)\right) = n m - \frac{m(m-1)}{2},
\end{align*} 
where $m = m_P(F)$, hence we get that 
\begin{align*}
    \dim_k \left( \mathfrak{m}^n/\mathfrak{m}^{n+1} \right) = m
\end{align*}
if $n \geq m_p(F)$. Now suppose $P$ is not a simple point, then $m_P(F) \geq 2$, and hence $\dim_k \mathfrak{m}/\mathfrak{m}^2 = 2$. Hence $\dim_k \mathfrak{m}/\mathfrak{m}^2 = 1$ implies $P$ is a simple point. Now if $P$ is a simple point then $m_P(F) = 1$, and hence we get that $\dim_k \mathfrak{m}/\mathfrak{m}^2 = m - \frac{m(m-1)}{2} = 1$, since $m=1$. Thus we have shown that $P$ is simple if and only if $\dim_k \mathfrak{m}/\mathfrak{m}^2 = 1$, and otherwise we have $\dim_k \mathfrak{m}/\mathfrak{m}^2 = 2$.
% Thus if $P$ is a simple point then we have $m_P(F) = 1$, hence $\dim_k \left(\mathfrak{m}/\mathfrak{m}^2\right) = $

\section{Problem \textcolor{maroon}{3.15}} %Problem 6

\ts{Part (a).} With out loss of generality let, $P=(0,0)$ and the corresponding maximal ideal in $k[x,y]$ is $\m_p =(x,y)$ and extension it's image in $\oo_p(\A^2)$ is $\m_p(\A^2)$. Now we know, $$k[x,y]/\m_p^n \simeq k[x,y]_{\m_p}/ \m_p^n k[x,y]_{\m_p} \simeq \oo_p/\m_p(\A^2)^n$$ (it follows from the fact, residue field/ localization commutes with quotienting). Enough to commute $\dim_k k[x,y]/\m_p^n$. Now, $\m_p^n = (x^n,x^{n-1}y,x^{n-2}y^2,\cdots,y^n)$. The basis of $k[x,y]/\m_p^n$ must be the standard $i$ forms, with $i<n$. For each $i$ there are such $i+1$ forms. And hence, $$\chi(n) = \dim_k \oo_p/\m_p(\A^2)^n = \dim_k k[x,y]/\m_p^n = \frac{n(n+1)}{2}$$

\vspace*{0.2cm}


\noindent \ts{Part (b).} Let, $\oo = \oo_p(\A^r)$ and $\m = \m_p(\A^r)$. Again let, $P=(0,\cdots,0)$ and $\m_p =(x_1,\cdots,x_r)$. Just by the similar past as above it is enough to calculate $\dim_k k[x_1,\cdots,x_r]/\m_p^n$. Now, $\m_p$ is generated by all standard forms of degree $n$. Thus,the basis of $k[x_1,\cdots,x_2]/\m_p^n$ must be the standard $i$ forms, with $i<n$. Thus the basis set can be written as, 
$$\mathcal{B} = \qty{x_1^{i_1}\cdots x_r^{i_r}:i_1+\cdots +i_r \leq n-1}$$ Now cardinality of the set is, \begin{align*}
    \abs{\mathcal{B}} &= \abs{\qty{x_1^{i_1}\cdots x_r^{i_r}:i_1+\cdots +i_r \leq n-1}} \\
    &= \abs{\qty{1^{i_0}x_1^{i_1}\cdots x_r^{i_r}:i_0+i_1+\cdots +i_r =n-1}} \\
    &= \binom{n+r-1}{r}
\end{align*}
So we must have,  $$\chi(n) = \dim_k \oo/\m^n= \dim_k k[x_1,\cdots,x_r]/\m_p^n = \binom{n+r-1}{r}= \frac{n(n+1)\cdots (n+r-1)}{r!}$$ Thus the leading coefficient is $1/r!$. $\hfill \blacksquare$

\section{Problem \textcolor{maroon}{3.16}} %Problem 7

In this problem we will try to trace the path of `Theorem 2' in `page 35'. Let, $\oo = \oo_P(V(F))$ and $P=(0,0,\cdots)$ and $\m = \m_p(V(F))$. Consider the maximal ideal $\m_p=(x_1,\cdots,x_r)$ corresponding to the point $P$. Let, $R = k[x_1,\cdots,x_r]$. Let, $m_P(F)=m$ (multiplicity of $P$ w.r.t $F$). Then we have the follows SES(short exact sequence) \[\begin{tikzcd}
    0 & {R/\m_p^{n-m}} & {R/\m_p^n} & {R/(F,\m_p^n)} & 0
    \arrow[from=1-1, to=1-2]
    \arrow["i", from=1-2, to=1-3]
    \arrow["\pi", from=1-3, to=1-4]
    \arrow[from=1-4, to=1-5]
    \end{tikzcd}\]
where $i$ is the map $i(\bar{G})=\overline{FG}$ and $\pi$ the natural projection map. It's an exact sequence. Thus by the previous problem we have, $$\dim_k R/(F,\m_p^n) = \dim_k R/\m_p^n - \dim_k R/\m_p^{n-m}= \binom{n+r-1}{r}-\binom{n+r-m-1}{r}$$ If we expand the above binomal coefficients it's not hard to see the above is polynomial over $n$, which has degree $r-1$ and leading coefficient is $m/r!$. Now from a rsult stated in class \textcolor{maroon}{*} it follows, $$R/(\m_p^n,F) \simeq \oo/\m^n$$ Thus $\chi(n) = \dim_k \oo/\m^n$ is a polynomial of $n$ of degree $(r-1)$ and leading coefficient is $m/r!$ as desired. $\hfill \blacksquare$


\section{Problem \textcolor{maroon}{3.23} and \textcolor{maroon}{3.24}} %last two problems 


\pagebreak

\section*{\S Exercises in chapter 2 needed for proving theorems in chapter 3}

\textcolor{maroon}{\ts{2.22}} We know given a map $f : V \to W$ between affine varieties, it extends to a ring homomorphism $f^{\ast}: \oo_{f(P)}(W) \to \oo_{P}(V)$. Now if we have an affine transformation $T : \A^n \to \A^n$ it will have inverse affine map $T^{-1}$. By the functoriality of pullback we can say they will induce $T^{\ast}$ and ${T^{-1}}^{\ast}$ in the corresponding local ring of regular functions. We can also note $T^{\ast} \circ {T^{-1}}^{\ast}$ and ${T^{-1}}^{\ast} \circ T^{\ast}$ is identity and hence $T^{\ast}$ is isomorphism. Thus $T^{\ast}: \oo_{T(P)}(\A^n) \to \oo_n(\A^n)$ is an isomorphism. If we restrict $T$ to $V \subset \A^k$ on that case $T$ will map $V$ to an isomorphic (as subvariety) copy $V^T\subset \A^n$. Again by the same computuation we can say, $\oo_{P}(V) \simeq \oo_{T(P)}(V^T)$ are isomorphic.

\vspace*{0.2cm}

\noindent  \textcolor{maroon}{\ts{2.34}} In this case if $F+G$ was reducible then we could write $F+G = fg$. Now if we homogenize the polynomial we will get, $$(F+G)^{\ast} = x_{n+1}F+G = f^{\ast} g^{\ast}$$ here treat $(F+G)^{\ast}$ as linear a polynomial over the ring $k[x_1,\cdots,x_n]$, which is UFD and hence by Gauss lemma $k[x_1,\cdots,x_n][x_{n+1}]$ is also UFD. But it can't have any non-constant factor over $k[x_1,\cdots,x_n][x_{n+1}]$. So, $F+G$ is irreducilbe. 

\vspace*{0.2cm}

\noindent \textcolor{maroon}{\ts{2.35(c),2.36}} is done in the computation step of \ts{3.15} part (b). So not doing it again.

\vspace*{0.2cm}

\noindent \textcolor{maroon}{\ts{2.44}*} (* marked in previous section) At first we will define a map $\psi : \oo_P(\A^n) \to \oo_P(V)/J'\oo_P(V)$. Firtly, we have the map $\oo_P(\A^n) \to \oo_P(V)$, which takes $f/g$ (such that $g(P)\neq 0$) to $\bar{f}/\bar{g}$ where $\bar{f},\bar{g}$ are $f,g$ modulo $I = I(V)$. It's not hard to see $g \notin I$ so $\bar{g}(P)\neq0$. Thus the map is well defined. $J$ is an ideal containing $I$ and $J'$ is the image in local ring, then there is a natural projection map $\oo_P(V)/J'\oo_p(V)$. Compositioon of this two map will be $\psi$.

\vspace*{0.2cm}

Now it's not hard to see $\psi$ is a surjective homomorphism. We will compute the kernal of it $\ker \psi$. Let, $f/g \in \oo_p(\A^n)$ such that $\bar{f}/\bar{g} \in J'\oo_p(V)$. We can write $$\bar{f}/\bar{g} = \sum \frac{j_i}{g'_i}$$ where $j_i \in J'$ and $g'_i$ are polynomial corresponding $g_i$(that don't vanish at $P$), i.e $g'_i =g_i \pmod{I}$. So, $\bar{f} \times \qty(\prod g'_i) \in J'\oo_p(V)$. Thus we can say, $f \times \qty(\prod g_i) \in J\oo_p(\A^n)$. Since $g_i$ are invertible we can say $f \in J\oo_p(\A^n)$. So, $\ker \psi \subseteq J\oo_p(\A^n)$. It's not hard to see $J\oo_p(\A^n) \subseteq \ker \psi$ thus we get, $\ker \psi = J\oo_p(\A^n)$. And thus we have a natural isomorphism $$\bar{\psi}: \oo_p(\A^n)/J\oo_p(\A^n) \to \oo_p(V)/J'\oo_p(V)$$  If $J=I$ then the right side is just $\oo_p(V)$ and thus $\oo_p(V)\simeq \oo_p(\A^n)/I\oo_p(\A^n)$.

\end{document}