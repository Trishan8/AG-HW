\documentclass[12pt]{article}
\usepackage{trishan2}

\title{\textbf{Assignment-4}}
\author{Trishan Mondal, Soumya Dasgupta, Aaratrick Basu}
\date{}

\begin{document}
\maketitle


\section{Problem \textcolor{maroon}{3.8}} %Problem 3

\ts{Part (a).} We will first prove it for the case when $P = Q =(0,0)$. In this case the polynomial map $T : \A^2 \to \A^2$ must look like $(f,g)$ where $f$ and $g$ are polynomials vanishing at $(0,0)$. In this case we can wite $f = f_i + \cdots + f_t$, where $f_i  \in k[x,y],i \geq 1$ is homogeneous polynomial of degree $i$. Similarly, we can write for $g$ (as both of them are vanishing at $(0,0)$). If $m = m_P(F)$ then $F = F_m +F_{m+1}\cdots$ again $F_i$ are homogeneous polynomial of degree $m$. Now $F^T = F(f,g)$'s lowest degree will come from $F_m(f,g)$ since both $f,g$ has at-least one degree term we can say, $m_Q(F^T)\geq m_p(F)$.  

\vspace*{0.2cm}

 Now we will use the fact proved in page (33) to prove it for any $P,Q$. Let, $Q\neq (0,0)$ or $Q = T(P)\neq (0,0)$. Let $T_1 : \A^2 \to \A^2$ be the affine transformation that maps $(0,0)$ to $Q$ and $T_2$ be the affine map sends $P$ to $(0,0)$. Note that $T_1  \circ T \circ T_2$ is a polynomial map and it maps $(0,0)$. So by the above calculation we can say, \begin{align*}
    m_P(F) &\leq m_{T_1  \circ T \circ T_2(0,0)}(F^{T_1  \circ T \circ T_2}) \\
    &= m_{T_1  \circ T (Q)}(F^{T_1  \circ T }) \text{ (By result of page 33})\\
    &= m_{T (Q)}(F^{T }) \text{ (By result of page 33})
 \end{align*}

 \vspace*{0.2cm}

 \noindent \ts{Part (b).} Again we will prove it for $P=Q=(0,0)$. Let $T = (f,g)$ and 
 $$
 J_Q T = \begin{pmatrix} \dfrac{\partial f}{\partial X}(Q) & \dfrac{\partial f}{\partial Y}(Q) \\ \dfrac{\partial g}{\partial X}(Q) & \dfrac{\partial g}{\partial Y}(Q) \end{pmatrix}.
 $$
 Assume $J_Q T$ is invertible. Since $J_Q T$ is invertible, we can't have both $\dfrac{\partial f}{\partial X}(Q) = 0$ and $\dfrac{\partial f}{\partial Y}(Q) = 0$ or both $\dfrac{\partial g}{\partial X}(Q) = 0$ and $\dfrac{\partial g}{\partial Y}(Q) = 0$. Again by similar computation of part $(a)$ we have, since $Q = (0,0)$, this implies that the decomposition of $f$ and $g$ into homogeneous polynomials are $f = f_1 + \cdots + f_m$ 
and $g = g_1 + \cdots + g_n$. Thus,
 $$
 F^T = F(f,g) = F_m(f,g) + F_{m+1}(f,g) + \ldots
 $$
 Since the lowest degree forms of $f$ and $g$ are of degree $1$, we have that $T$ does not decrease the degree of the form $F_m(f,g)$. Similarly, $T$ does not decrease the degree of $F_{m+1}(f,g), \cdots$. Therefore we have that $m_{(0,0)}(F^T) = m_{(0,0)}(F)$. Now assume that either $Q = (a_1,b_1) \neq (0,0)$ or $P = (a_2,b_2) \neq (0,0)$. Assume that $J_Q T$ is invertible. Let $T_1$  be the translation that takes $(0,0)$ to $Q$ and $T_2$ be the translation that takes $P$ to $(0,0)$. Then we have
 $$
  d(T_1\circ T\circ T_2) = d(T_1)\circ d(T) \circ d(T_2)
 $$
is also invertible. By the previous case $m_P(F) = m_{(0,0)}(F^{T_1\circ T\circ T_2})$and the similar computation of multiplicities we can say $m_P(F^T)=m_Q(F)$. And hence our proof is complete. 

\vspace*{0.2cm}

\noindent \ts{Part (c).} If $F= Y-X^2$ and $T =(X^2,Y)$, $P=Q =(0,0)$ we can see $m_P(F^T)=m_P(Y-X^4)=m_Q(F)=m_Q(Y-X^2)$. But the jacobian of $T$ is not invertible at $(0,0)$,as it is given by the matrix$\begin{pmatrix}
    0 & 0\\
    0 & 1
\end{pmatrix}$.\Qed

\section{Problem \textcolor{maroon}{3.15}} %Problem 6

\ts{Part (a).} With out loss of generality let, $P=(0,0)$ and the corresponding maximal ideal in $k[x,y]$ is $\m_p =(x,y)$ and extension it's image in $\oo_p(\A^2)$ is $\m_p(\A^2)$. Now we know, $$k[x,y]/\m_p^n \simeq k[x,y]_{\m_p}/ \m_p^n k[x,y]_{\m_p} \simeq \oo_p/\m_p(\A^2)^n$$ (it follows from the fact, residue field/ localization commutes with quotienting). Enough to commute $\dim_k k[x,y]/\m_p^n$. Now, $\m_p^n = (x^n,x^{n-1}y,x^{n-2}y^2,\cdots,y^n)$. The basis of $k[x,y]/\m_p^n$ must be the standard $i$ forms, with $i<n$. For each $i$ there are such $i+1$ forms. And hence, $$\chi(n) = \dim_k \oo_p/\m_p(\A^2)^n = \dim_k k[x,y]/\m_p^n = \frac{n(n+1)}{2}$$

\vspace*{0.2cm}


\noindent \ts{Part (b).} Let, $\oo = \oo_p(\A^r)$ and $\m = \m_p(\A^r)$. Again let, $P=(0,\cdots,0)$ and $\m_p =(x_1,\cdots,x_r)$. Just by the similar past as above it is enough to calculate $\dim_k k[x_1,\cdots,x_r]/\m_p^n$. Now, $\m_p$ is generated by all standard forms of degree $n$. Thus,the basis of $k[x_1,\cdots,x_2]/\m_p^n$ must be the standard $i$ forms, with $i<n$. Thus the basis set can be written as, 
$$\mathcal{B} = \qty{x_1^{i_1}\cdots x_r^{i_r}:i_1+\cdots +i_r \leq n-1}$$ Now cardinality of the set is, \begin{align*}
    \abs{\mathcal{B}} &= \abs{\qty{x_1^{i_1}\cdots x_r^{i_r}:i_1+\cdots +i_r \leq n-1}} \\
    &= \abs{\qty{1^{i_0}x_1^{i_1}\cdots x_r^{i_r}:i_0+i_1+\cdots +i_r =n-1}} \\
    &= \binom{n+r-1}{r}
\end{align*}
So we must have,  $$\chi(n) = \dim_k \oo/\m^n= \dim_k k[x_1,\cdots,x_r]/\m_p^n = \binom{n+r-1}{r}= \frac{n(n+1)\cdots (n+r-1)}{r!}$$ Thus the leading coefficient is $1/r!$. $\hfill \blacksquare$

\section{Problem \textcolor{maroon}{3.16}} %Problem 7

In this problem we will try to trace the path of `Theorem 2' in `page 35'. Let, $\oo = \oo_P(V(F))$ and $P=(0,0,\cdots)$ and $\m = \m_p(V(F))$. Consider the maximal ideal $\m_p=(x_1,\cdots,x_r)$ corresponding to the point $P$. Let, $R = k[x_1,\cdots,x_r]$. Let, $m_P(F)=m$ (multiplicity of $P$ w.r.t $F$). Then we have the follows SES(short exact sequence) \[\begin{tikzcd}
    0 & {R/\m_p^{n-m}} & {R/\m_p^n} & {R/(F,\m_p^n)} & 0
    \arrow[from=1-1, to=1-2]
    \arrow["i", from=1-2, to=1-3]
    \arrow["\pi", from=1-3, to=1-4]
    \arrow[from=1-4, to=1-5]
    \end{tikzcd}\]
where $i$ is the map $i(\bar{G})=\overline{FG}$ and $\pi$ the natural projection map. It's an exact sequence. Thus by the previous problem we have, $$\dim_k R/(F,\m_p^n) = \dim_k R/\m_p^n - \dim_k R/\m_p^{n-m}= \binom{n+r-1}{r}-\binom{n+r-m-1}{r}$$ If we expand the above binomal coefficients it's not hard to see the above is polynomial over $n$, which has degree $r-1$ and leading coefficient is $m/r!$. Now from a rsult stated in class it follows, $$R/(\m_p^n,F) \simeq \oo/\m^n$$ Thus $\chi(n) = \dim_k \oo/\m^n$ is a polynomial of $n$ of degree $(r-1)$ and leading coefficient is $m/r!$ as desired. $\hfill \blacksquare$


\section{Problem \textcolor{maroon}{3.23} and \textcolor{maroon}{3.24}} %last two problems 



\end{document}