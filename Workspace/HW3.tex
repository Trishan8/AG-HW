\documentclass[12pt]{article}
\usepackage{trishan2}
\usepackage{ebgaramond}
\title{Assignment-3}
\author{Trishan Mondal, Soumya Dasgupta, Aaratrick Basu}
\date{}

\begin{document}
\maketitle
\setcounter{section}{3}

\begin{prob} %trishan 1
    (UAG 5.1) A rgular function on $\mathbb{P}^1$ is constant. Deduce that there are no non-constant morphisms $\mathbb{P}^1 \to \A^m$ for $m \geq 1$.
\end{prob}

\sol. Suppose  $f \in k(\mathbb{P}^1)$ be a rational function, which is regular everywhere. If we restrict it to the affine piece $\A_{(0)}$, we get $f(x,1) = p(x) \in k[x]$ (as for the case of affine variety $\operatorname{dom} f = V$ iff $f \in k[V]$). Similarly, we can restrict $f$ to another affine piece $\A_{\infty}$. We get, $f(1,y) = f(1/y,1) = p(1/y) \in k[y]$. It is possible iff $p$ is constant.

\vspace*{0.2cm}

\noindent Any morphisms  $\mathbb{P}^1 \to \A^m$ can be given by $(f_1,\cdots,f_m)$ where $f_i$ are regular on $\mathbb{P}^1$. Thus the function $f$ is constant by the previous part. \Qed

% problem 5.2 (complete)
\begin{prob}
    (\textit{The quadric surface in $\mathbb{P}^3$}).
    \begin{enumerate}
        \item[(i)] Show that the Segre embedding of $\mathbb{P}^1 \times \mathbb{P}^1$ gives an isomorphism of $\mathbb{P}^1 \times \mathbb{P}^1$ with the quadric
            \begin{align*}
                S_{1,1} = Q : (X_0X_3 = X_1X_2) \subseteq \mathbb{P}^3.
            \end{align*}
        \item[(ii)] What are the images in $Q$ of the two families of lines $\{p\} \times \mathbb{P}^1$ and $\mathbb{P}^1 \times \{p\}$ in $\mathbb{P}^1 \times \mathbb{P}^1$? Use this to find some disjoint lines in $\mathbb{P}^1 \times \mathbb{P}^1$, and conclude from this that $\mathbb{P}^1 \times \mathbb{P}^1 \not\cong \mathbb{P}^2$.
        \item[(iii)] Show that there are two lines of $Q$ passing through the point $P = (1,0,0,0)$ and that the complement $U$ of these two lines is the image of $\mathbb{A}^1 \times \mathbb{A}^1$ under the Segree embedding.
        \item[(iv)] Show that under the projection $\pi\vert_{Q} : Q \dashrightarrow \mathbb{P}^2$, $U$ maps isomorphically to a copy of $\mathbb{A}^2$, and the two lines through $P$ are mapped to two points of $\mathbb{P}^2$.
        \item[(v)] Find $\mathrm{dom} \pi$ and $\mathrm{dom} \varphi$, and give a geometric interpretation of the singularities of $\pi$ and $\varphi$.
    \end{enumerate}
\end{prob}

\sol.
\begin{enumerate}
    \item[(i)] Let $\varphi : \mathbb{P}^1 \times \mathbb{P}^1 \to \mathbb{P}^3$, $([X_0,X_1],[Y_0,Y_1]) \mapsto [X_0Y_0, X_0Y_1, X_1Y_0, X_1Y_1]$ be the Segree embedding. Then we clearly have $\im \varphi = S_{1,1} \subseteq Q$. Since we know that the Segree embedding $S_{1,1} \cong \mathbb{P}^1 \times \mathbb{P}^1$. Its enough to show that $Q \subseteq S_{1,1}$. Note that
        \begin{align*}
            Q & = \{ [X_0,X_1,X_2,X_3] \in \mathbb{P}^3 \mid X_0X_3 - X_1X_2 = 0 \}                                                             \\
              & = \left\{ [X_0,X_1,X_2,X_3] \in \mathbb{P}^3 \mid \det\begin{pmatrix} X_0 & X_1 \\ X_2 & X_3 \end{pmatrix} = 0\right\}          \\
              & = \left\{ [X_0,X_1,X_2,X_3] \in \mathbb{P}^3 \mid \mathrm{rk}\begin{pmatrix} X_0 & X_1 \\ X_2 & X_3 \end{pmatrix} = 1 \right\},
        \end{align*}
        the rank can not be zero, as at least one of the entries $X_0, X_1, X_2, X_3$ is nonzero. Let $[X_0,X_1,X_2,X_3] \in Q$, and WLOG assume $X_0 \neq 0$, then we get there exists $\lambda, \mu \neq 0$ such that
        \begin{align*}
            \begin{pmatrix}
                X_0 \\ X_2
            \end{pmatrix} = \lambda \begin{pmatrix}
                                        X_1 \\ X_3
                                    \end{pmatrix}  \quad \mbox{ and } \quad \begin{pmatrix}
                                                                                X_0 \\ X_1
                                                                            \end{pmatrix} = \mu \begin{pmatrix}
                                                                                                    X_2 \\ X_3
                                                                                                \end{pmatrix}
        \end{align*}
        Thus we get that $X_1 = \frac{X_0}{\lambda}$, $X_2 = \frac{X_0}{\mu}$ and $X_3 = \frac{X_2}{\lambda} = \frac{X_0}{\mu\lambda}$, thus we get that
        \begin{align*}
            [X_0,X_1,X_2,X_3] = \left[ X_0, \frac{X_0}{\lambda}, \frac{X_0}{\mu}, \frac{X_0}{\mu\lambda} \right] = [\mu\lambda,\mu,\lambda,1] = \varphi([\mu,1],[\lambda,1]).
        \end{align*}
        Therefore we have proved that $Q \subseteq S_{1,1}$, hence we get that $\varphi$ induces an isomorphism of $S_{1,1}$ and $Q$.

    \item[(ii)] We have $\varphi(\{p\} \times \mathbb{P}^1) = \{ [aY_0, aY_1, bY_0, bY_1] \mid [Y_0,Y_1] \in \mathbb{P}^1 \}$, which is equation of the line passing through $[a,0,b,0],[0,a,0,b] \in \mathbb{P}^3$. Similarly image of $\mathbb{P}^1 \times \{p\}$ is again a line in $\mathbb{P}^3$. But then note that for $p \neq q \in \mathbb{P}^1$, we have $(\{p\} \times \mathbb{P}^1) \cap (\{q\} \times \mathbb{P}^1) = \emptyset$, hence their images are disjoint lines in $Q$. But we know that any two lines in $\mathbb{P}^2$ have a intersection, hence $\mathbb{P}^1 \times \mathbb{P}^1 \not\cong \mathbb{P}^2$.

    \item[(iii)] Let us consider the image of $\mathbb{A}^1 \times \mathbb{A}^1$ in $\mathbb{P}^3$ under the Segre embedding. We get
        \begin{align*}
            \varphi(\mathbb{A}^1 \times \mathbb{A}^1) = \{ [ab,a,b,1] \in \mathbb{P}^3 \mid a,b \in k \}.
        \end{align*}
        Now consider the two lines $\ell_1 = \{[\mu,0,\lambda,0] \in \mathbb{P}^3 \mid [\mu,\lambda] \in \mathbb{P}^1 \}$ and $\ell_2 = \{[\mu,\lambda,0,0] \in \mathbb{P}^3 \mid [\mu,\lambda] \in \mathbb{P}^1 \}$ through $[1,0,0,0]$ and contained in $Q$. We claim that the complement $U$ of these two lines is $\varphi(\mathbb{A}^1 \times \mathbb{A}^1)$. Clearly we have $\varphi(\mathbb{A}^1 \times \mathbb{A}^1) \cap (\ell_1 \cup \ell_2) = \emptyset$. Conversely let $[X_0,X_1,X_2,X_3] \notin \varphi(\mathbb{A}^1 \times \mathbb{A}^1)$, then $[X_0,X_1,X_2,X_3] = \varphi([a,b],[1,0]) = [a,0,b,0] \in \ell_1$ or $[X_0,X_1,X_2,X_3] = \varphi([1,0],[c,d]) = [c,d,0,0] \in \ell_2$. Therefore we have shown that $U = \varphi(\mathbb{A}^1 \times \mathbb{A}^1)$.

    \item[(iv)] Under the projection $\pi\vert_Q : Q \dashrightarrow \mathbb{P}^2$, $[X_0,X_1,X_2,X_3] \mapsto [X_1,X_2,X_3]$. Then
        \begin{align*}
            \pi(U) = \pi(\varphi(\mathbb{A}^1 \times \mathbb{A}^1)) = [a,b,1] \in \mathbb{A}^2 \subseteq \mathbb{P}^2.
        \end{align*}
        And the two lines $\ell_1$ and $\ell_2$ maps to the two points $[0,1,0]$ and $[1,0,0]$ respectively.

    \item[(v)] Since $\pi$ is just the projection of $\mathbb{P}^3$ from the point $[0,0,0,1]$ onto the $\mathbb{P}^2$, its domain is given by $\mathrm{dom}\, \pi = \mathbb{P}^3 \setminus [0,0,0,1]$, and hence $\mathrm{dom}\, \pi\vert_Q = Q \setminus [0,0,0,1]$. On the other hand the domain of the Segre embedding is $\mathrm{dom} \, \varphi = \mathbb{P}^1 \times \mathbb{P}^1$.
\end{enumerate}

\begin{prob}
    Which of the following expressions define rational maps \( \varphi: \mathbbm{P}^n \to \mathbbm{P}^m \) (with \( n,m = 1 \) or 2) between projective spaces of appropriate dimensions? In each case determine \( \operatorname{dom} \varphi, \) say if \( \varphi \) is birational, and if so, describe the inverse map.
\end{prob}

\sol.
\begin{enumerate}[label = (\alph*)]
    \item The given map is a rational map. This is because it is well-defined for all \( [x,y,z] \in \mathbbm{P}^2 \setminus \qty{[0,0,1]} \) and is a rational function in each coordinate of the image. We therefore have
          \[
              \operatorname{dom} \varphi =[x,y,z] \in \mathbbm{P}^2 \setminus \qty{[0,0,1]}.
          \]
          Further, this is a birational map, as it has the rational inverse given by the map in (c), \( [x,y] \mapsto [x,y,0] \).
          \smallskip
    \item The given map is not a rational map. This is because
          \[
              \varphi([1,0]) = [1,0,1] \neq [2,0,1] = \varphi([2,0]),
          \]
          but \( [1,0] \neq [2,0] \).
          \smallskip

    \item The given map is a rational map. This is because it is well-defined for all \( [z,y] \in \mathbbm{P}^1 \) and is a rational function in each coordinate of the image. We therefore have
          \[
              \operatorname{dom} \varphi = \mathbbm{P}^1.
          \]
          Further, this is a birational map, as it has the rational inverse given by the map in (a), \( [x,y,z] \mapsto [x,y] \).
          \smallskip

    \item The given map is a rational map. This is because it is well-defined for all \( [x,y,z] \in \mathbbm{P}^2 \) with \( xyz \neq 0 \), and is a rational function in each coordinate of the image. We therefore have,
          \[
              \operatorname{dom}  \varphi = \qty{[x,y,z] \mid xyz \neq 0}.
          \]
          Further, \( \varphi^2 \) is the identity map on \( \operatorname{dom} \varphi \), and so it is a birational map.
          \smallskip

    \item The given map is a rational map. This is because it is well-defined for all \( [x,y,z] \in \mathbbm{P}^2 \) with \( z \neq 0 \), and is a rational function in each coordinate of the image. We therefore have,
          \[
              \operatorname{dom} \varphi = \qty{[x,y,z] \mid z \neq 0}.
          \]
          The map is not birational as the function fields of the domain and image are not isomorphic.
          \smallskip

    \item The given map is a rational map. This is because it is well-defined for all \( [x,y,z] \in \mathbbm{P}^2 \) with one of \( x,y \) non-zero, and is a rational function in each coordinate of the image. We therefore have,
          \[
              \operatorname{dom} \varphi = \mathbbm{P}^2 \setminus \qty{[0,0,1]}.
          \]
          The map is not birational as there is no rational inverse.
\end{enumerate}

% problem 5.6 (complete)
\begin{prob}
    Let $C \subseteq \mathbb{P}^3$ be an irreducible curve defined by $C = Q_1 \cap Q_2$, where $Q_1 : (TX = q_1),$ and $Q_2 : (TY = q_2)$, with $q_1,q_2$ quadratic forms in $X,Y,Z$. Show that the projection $\pi : \mathbb{P}^3  \dashrightarrow \mathbb{P}^2$ defined by $(X,Y,Z,T) \mapsto (X,Y,Z)$ restricts to an isomorphism of $C$ with the plane curve $D \subseteq \mathbb{P}^2$ given by $Xq_2 = Yq_1$.
\end{prob}

\sol.
Let us define the map $\varphi : D \dashrightarrow C$, as follows,
\begin{align*}
    [X,Y,Z] \mapsto \begin{cases} [X,Y,Z,\frac{q_1}{X}] & \mbox{ if } X \neq 0 \\
              [X,Y,Z,\frac{q_2}{Y}] & \mbox{ if } Y \neq 0\end{cases}
\end{align*}
Note that this is indeed a map from $D$ to $C$, as if $[X,Y,Z] \in D$ with $X \neq 0$, then we get that $Xq_2 = Yq_1$, and hence, $TX = q_1$ and $TY = \frac{Yq_1}{X} = \frac{Xq_2}{X} = q_2$, thus $\varphi([X,Y,Z]) \in C$, and similarly for $Y \neq 0$, we have $[X,Y,Z,T] = \varphi([X,Y,Z]) \in C$. On the other hand restricting the projection onto $C$, we get that $\pi([X,Y,Z,T]) = [X,Y,Z]$, and since $TX = q_1$ and $TY = q_2 $ we get that $Yq_1 = TXY = Xq_2$, thus we indeed have $[X,Y,Z] \in D$.

Finally note that $\pi\vert_C \circ \varphi = \mathrm{id}_D$ is obvious and
\begin{align*}
    \varphi(\pi\vert_C([X,Y,Z,T])) = \varphi([X,Y,Z]) = \begin{cases}
                                                            [X,Y,Z,\frac{q_1}{X}] & \mbox{ if } X \neq 0 \\
                                                            [X,Y,Z,\frac{q_2}{Y}] & \mbox{ if } Y \neq 0
                                                        \end{cases} = [X,Y,Z,T],
\end{align*}
where the last equality follows from the fact that $TX = q_1$ and $TY = q_2$ for points in $C$. Thus we indeed have $\varphi \circ \pi\vert_C = \mathrm{id}_C$. Hence $\pi$ restricted onto $C$ induces an isomorphism of $C$ with the plane curve $D$.

\begin{prob}
    For each of the following plane curves, write down the 3 standard affine pieces, and determine the intersection of the curve with the 3 coordinate axes.
    \begin{enumerate}[label = (\alph*)]
        \item \( y^2z = x^3 + axz^2 + bz^3 \)
        \item \( x^2y^2 + y^2z^2 + x^2z^2 = 2xyz(x+y+z) \)
        \item \( xz^3=(x^2+z^2)y^2 \)
    \end{enumerate}
\end{prob}
\sol. \begin{enumerate}[label = (\alph*)]
    \item The affine pieces are:
          \begin{enumerate}[label = (\roman*)]
              \item \( (x=1): y^2z = 1+az^2+bz^3 \)
              \item \( (y=1): z = x^3+axz^2+bz^3 \)
              \item \( (z=1): y^2=x^3+ax+b \)
          \end{enumerate}

          The intersections with the coordinate axes are:
          \begin{enumerate}[label = (\roman*)]
              \item \( x- \)axis: \( x^3=0 \)
              \item The intersection with the \( y- \)axis, is the complete axis, as the equation of the curve holds trivially when \( x=z=0 \).
              \item \( z- \)axis: \( z^3=0 \)
          \end{enumerate}
          \smallskip

    \item The affine pieces are:
          \begin{enumerate}[label = (\roman*)]
              \item \( (x=1): y^2z^2 + (y-z)^2-2yz(y+z)=0 \)
              \item \( (y=1): z^2x^2 + (z-x)^2-2zx(z+x)=0 \)
              \item \( (z=1): x^2y^2 + (x-y)^2-2xy(x+y)=0 \)
          \end{enumerate}
          The intersection of the given curve with any of the three axes is the complete axis in each case, because setting any two variables to 0 forces the equation of the curve to hold trivially.
          \smallskip

    \item The affine pieces are:
          \begin{enumerate}[label = (\roman*)]
              \item \( (x=1): z^3=(1+z^2)y^2 \)
              \item \( (y=1): xz^3=x^2+z^2 \)
              \item \( (z=1): x=(x^2+1)y^2 \)
          \end{enumerate}
          The intersection of the given curve with any of the three axes is the complete axis in each case, because setting any two variables to 0 forces the equation of the curve to hold trivially.
\end{enumerate}

\begin{prob} %trishan 2
    (UAG 5.7) Let $\varphi : \mathbb{P}^1 \to \mathbb{P}^1$ be an isomorphism; identify graph of $\varphi$ as subvariety of $\mathbb{P}^1 \times \mathbb{P}^1\subset \mathbb{P}^3$. Now do the same if $\varphi : \Pn^1 \to \Pn^1$ is given by map $(X,Y)\mapsto (X^2,Y^2)$.
\end{prob}

\sol. Consider the identity map $\id : \Pn^1 \to \Pn^1$ and the given isomorphism, it will give us a map $\id \times \varphi :\Pn^1 \times \Pn^1 \to \Pn^1\times \Pn^1$ by $(x,y)\mapsto (x,\varphi(x))$. Under the identification of $\Pn^1 \times \Pn^1 = \Pn^3$ we can say, $\id \times \varphi$ is also a morphism of variety. In the variety $\Pn^1 \times \Pn^1$, the diagonal $\D = \qty{(x,x): x \in \Pn^1}$ is closed (simply because it is given by the vanishing of $x_0-x_2$ and $x_1-x_3$ where $[x_0:x_1]$ and $[x_2:x_3]$ are co-ordinates of two copies of $\Pn^1$). It's not hard to see the graph of $\varphi$ is given by the inverse image of $\D$ under $\id \times \varphi$. $$\Gamma(\varphi)= (\id \times \varphi)^{- 1}(\D)$$ Since the graph is closed it's inverse image will also be closed. Thus the graph is a closed set and under zariski topology any closed set is given by vanishing of some set of polynomials. This will help us to identify $\Gamma(\varphi)$ as a subvariety of $\Pn^1\times \Pn^1$. If $\varphi$ is given by $[x:y]\to [f(x,y):g(x,y)]$ then the graph can be given by the image of following vanishing set under segre embedding $$\qty{[x_0:x_1:x_2:x_3]: x_2 =f(x_0,x_1), x_3=g(x_0,x_1)}$$

\vspace*{0.2cm}

\noindent If, $\varphi$ given by $[x,y]\mapsto [x^2:y^2]$ the image of $([x:y],[x^2,y^2])$ is $[x^3:xy^2:yx^2:y^3]$(image under segre embedding). Which is rational curve $\Pn^1 \to \Pn^3$, a sub-variety of $\Pn^3$.
$$\Gamma(\varphi)\simeq \text{Rational curve in }\Pn^3$$


\begin{prob}
    \begin{enumerate}
        \item[(i)] Prove that the product of two irreducible algebraic sets is again irreducible.
        \item[(ii)] Describe the closed sets of the topology on $\mathbb{A}^2 = \mathbb{A}^1 \times \mathbb{A}^1$ which is the product of the Zariski topologies on the two factors; now find a closed subset of the Zariski topology of $\mathbb{A}^2$ not of this form.
    \end{enumerate}
\end{prob}

\sol.
\begin{enumerate}
    \item[(i)] Suppose that $X \times Y = Q_1 \cup Q_2$, with each $Q_i$ a closed subset of $X \times Y$.
        For each $x \in X$, the closed set $\{x\} \times Y$ is isomorphic to $Y$, and is therefore irreducible.
        Since $\{x\} \times Y= ((\{x\} \times Y) \cap Q_1) \cup ((\{x\} \times Y ) \cap Q_2)$ either $\{x\} \times Y \subseteq Q_1$ or else $\{x\} \times Y \subseteq Q_2$.

        The subset $X_1 \subseteq X$ consisting of those $x \in X$ with $\{x\} \times Y \subseteq Q_1$ is a closed subset, to see this note that $X_1 = \cap_{y \in Y} X_y$, where $X_y$ is the collection of points $x \in X$ such that $\{x\} \times \{y\} \in Q_1$. Since $X_y \times \{y\} = (X \times \{y\}) \cap Q_1$, $X_y$ and hence $X_1$ is closed. Similarly we can define the closed subset $X_2$.

        Since $X = X_1 \cup X_2$ and $X$ is irreducible, we either have $X = X_1$ or $X = X_2$. But $X = X_i$ implies $X \times Y = Q_i$, contradicting the fact the both of the $Q_i$'s are nonempty. Therefore $X \times Y$ is irreducible.

    \item[(ii)] We know that the closed subsets of $\mathbb{A}^1$ under the Zariski topology are finite subsets of $\mathbb{A}^1$ and the whole set $\mathbb{A}^1$. Thus under the product topology on $\mathbb{A}^2 = \mathbb{A}^1 \times \mathbb{A}^1$ closed subsets are once again finite subsets of $\mathbb{A}^1 \times \mathbb{A}^1$, $\{x_1,\dots,x_n\} \times \mathbb{A}^1$, $\mathbb{A}^1 \times \{y_1,\dots,y_m\}$ and $\mathbb{A}^1 \times \mathbb{A}^1$.

        Consider the closed subset $C = V(X-Y) = \{(a,a) \mid a \in k\} \subseteq \mathbb{A}^2$. If $k$ is an infinite field, then $C$ does not belong to any of the closed sets coming from the product topology on $\mathbb{A}^1 \times \mathbb{A}^1$.
\end{enumerate}

\begin{prob}
    Let \( C \) be the cubic curve of (5.0). Prove that any regular function on \( C \) is constant.
\end{prob}

\sol. The given curve is \( C: (Y^2Z = X^3 + aXZ^2 + bZ^3) \subset \mathbbm{P}^2 \). The affine pieces are
\[
    C_{(0)}: y^2 = x^3+ax+b, \quad C_{(\infty)}: z' = x'^3+ax'z'^2+bz'^3
\]

Let \( f \) be a regular function on \( C \). Then, \( \operatorname{dom} f \supset C_{(0)} \), and so, \( f \in k[C_{(0)}] = k[x,y]/(y^2-x^3-ax-b) \). Hence, there is \( q,r \in k[x] \) such that \( f(x,y) \equiv q(x) + y r(x) \) in \( k[C_{(0)}] \). Now, as \( \operatorname{dom} f \supset C_{(\infty)} \), we get that
\[
    q \qty(\frac{x'}{z'}) + \frac{1}{z'}r\qty(\frac{x'}{z'}) \equiv p(x',z'),
\]
for some polynomial \( p \). Therefore, we can multiply out the denominators to get an expression
\[
    \widetilde{q}(x',z') + \widetilde{r}(x',z') = p(x',z')z'^m + A(x',z')g,
\]
in \( k[x',z'] \), where \( \widetilde{q} \) is homogeneous of degree \( m \), \( \widetilde{r} \) is homogeneous of degree \( m-1 \), \( g = x'^3+ax'z'^2+bz'^3-z' \). We now write \( p = p_1+p_2 \) and \( A = A_1+A_2 \), where \( p_1,A_1 \) consist of the odd degree terms and \( p_2,A_2 \) consist of the even degree terms. Then, assuming \( m \) is odd, we get
\[
    \widetilde{q} = p_2z'^m + A_1g, \quad \widetilde{r} = p_1z_1^m + A_2g.
\]
A similar expression holds in case \( m \) is even, by switching \( p_1 \) with \( p_2 \) and \( A_1 \) with \( A_2 \). Now, \( \widetilde{q} \) is homogeneous of degree \( m \), and hence, \( A_1g \) must have degree at least \( m \). Therefore, we get (as \( g \) has the term \( z' \)) that \( z' \mid \widetilde{q}\). Similarly, \( z' \mid \widetilde{r} \). Hence, we can divide the entire expression by \( z' \), and get \( \widetilde{q} \) homogeneous of degree \( m-1 \) and \( \widetilde{r} \) homogeneous of degree \( m-2 \). Hence, assuming that \( m \) is the least possible we get \( m=0 \), and so, \( f \equiv c \) for some constant \( c \). This shows that \( f \) must in fact be constant, as was required. \(\hfill \blacksquare\)

\begin{prob} %trishan 3
    (UAG 5.13) Study the embedding $\varphi : \Pn^2 \to \Pn^5$ given by $[x:y:z] \mapsto [x^2:xy:xz:y^2:yz:z^2]$ and prove that $\varphi$ is an isomorphism. Prove that the lines of $\Pn^2$ go over the conics of $\Pn^5$ and the conics go over the twisted quartics of $\Pn^5$.

    \vspace*{0.2cm}

    \noindent For any line $\ell \subset \Pn^2$, write $\pi(\ell) \subseteq \Pn^5$ for the projective plane spanned by the conics $\varphi(\ell)$. Prove that union of $\pi(\ell)$ taken over all $\ell \subset \Pn^2$ is a cubic hypersurface $\Sigma \subseteq \Pn^5$.
\end{prob}

\sol. Consider the following vanishing set on $\Pn^5$, $$S = V(t_0t_3-t_1^2, t_3t_5-t_4^2,t_0t_5-t_2^2,t_1t_2-t_0t_4,t_1t_4-t_3t_2,t_2t_4-t_5t_1)$$
It's not hard to see $\im \varphi \subset S$. Now note that the map $\varphi$ gives us a surjective map between the following vector spaces, \[
    \qty{\text{homogeneous quadratic polynomials in }t_0,\cdots,t_5} \to  \qty{\text{homogeneous quartics in } x,y,z}
\]
The first V.S is of dimension $21$ and the later one has dimension $15$. Thus the kernal has dimension $6$. Now note that the polynomials defining $S$ are linearly independent. So, $\im \varphi = S$. Thus the image of $\varphi$ is given by the variety $S$. Now take the map $\psi : S \to \Pn^3$ that maps $[t_0:\cdots:t_5] \to [t_0:t_1:t_2]$ works as the inverse map of $\varphi$ (it is defined except for $[0:0:0:0:0:1]$). So, $\varphi$ is an isomorphism.  Any line in $\Pn^2$ can be given by the set $\qty{[X:Y:AX+BY]}$, the image of that under $\varphi$ is $(X^2, XY, AX^2+BXY, Y^2, AXY + BY^2, A^2X^2 + 2AXBY + B^2Y^2)$. Note that the projective transformation given by
\[
    \begin{bmatrix}
        1    & 0    & 0 & 0    & 0 & 0 \\
        0    & 1    & 0 & 0    & 0 & 0 \\
        -A   & -B   & 1 & 0    & 0 & 0 \\
        0    & 0    & 0 & 1    & 0 & 0 \\
        0    & A    & 0 & -B   & 1 & 0 \\
        -A^2 & -2AB & 0 & -B^2 & 0 & 1 \\
    \end{bmatrix}
\]
is valid since its determinant is 1 (easily computed using the fact that it is a lower triangular matrix). Any conic in $\Pn^2$ can be re-parametrized so that it is given by $[u^2:uv:v^2]$. It's image in $S$ is twisted quardics.
\vspace*{0.2cm}

\noindent To do the last part we can also identify $S$ as the following set,\[S = \qty{ [t_0:t_1:\cdots:t_5] \in \Pn^5:\operatorname{rank }\begin{pmatrix}
            t_0 & t_1 & t_2 \\
            t_1 & t_3 & t_4 \\
            t_2 & t_4 & t_5
        \end{pmatrix} \leq 1}\] From the above identification of $S$ we can say, $\cup_{\ell \subset \Pn^2}\pi(\ell)$ is given by $\det \begin{pmatrix}
        t_0 & t_1 & t_2 \\
        t_1 & t_3 & t_4 \\
        t_2 & t_4 & t_5
    \end{pmatrix}=0$. This clearly determines a hyper-surface in $\Pn^5$. $\hfill \blacksquare$

\end{document}
