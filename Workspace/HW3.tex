\documentclass[12pt]{article}
\usepackage{trishan2}
\usepackage{ebgaramond}
\title{Assignment-3}
\author{Trishan Mondal, Soumya Dasgupta, Aaratrick Basu}
\date{}

\begin{document}
 \maketitle
 \setcounter{section}{3}

 \begin{prob} %trishan 1
  (UAG 5.1) A rgular function on $\mathbb{P}^1$ is constant. Deduce that there are no non-constant morphisms $\mathbb{P}^1 \to \A^m$ for $m \geq 1$.
 \end{prob}

\sol. Suppose  $f \in k(\mathbb{P}^1)$ be a rational function, which is regular everywhere. If we restrict it to the affine piece $\A_{(0)}$, we get $f(x,1) = p(x) \in k[x]$ (as for the case of affine variety $\operatorname{dom} f = V$ iff $f \in k[V]$). Similarly, we can restrict $f$ to another affine piece $\A_{\infty}$. We get, $f(1,y) = f(1/y,1) = p(1/y) \in k[y]$. It is possible iff $p$ is constant.

\vspace*{0.2cm}

\noindent Any morphisms  $\mathbb{P}^1 \to \A^m$ can be given by $(f_1,\cdots,f_m)$ where $f_i$ are regular on $\mathbb{P}^1$. Thus the function $f$ is constant by the previous part. \Qed

\begin{prob} %trishan 2
    (UAG 5.7) Let $\varphi : \mathbb{P}^1 \to \mathbb{P}^1$ be an isomorphism; identify graph of $\varphi$ as subvariety of $\mathbb{P}^1 \times \mathbb{P}^1\subset \mathbb{P}^3$. Now do the same if $\varphi : \Pn^1 \to \Pn^1$ is given by map $(X,Y)\mapsto (X^2,Y^2)$.
\end{prob} 

\sol. Consider the identity map $\id : \Pn^1 \to \Pn^1$ and the given isomorphism, it will give us a map $\id \times \varphi :\Pn^1 \times \Pn^1 \to \Pn^1\times \Pn^1$ by $(x,y)\mapsto (x,\varphi(x))$. Under the identification of $\Pn^1 \times \Pn^1 = \Pn^3$ we can say, $\id \times \varphi$ is also a morphism of variety. In the variety $\Pn^1 \times \Pn^1$, the diagonal $\D = \qty{(x,x): x \in \Pn^1}$ is closed (simply because it is given by the vanishing of $x_0-x_2$ and $x_1-x_3$ where $[x_0:x_1]$ and $[x_2:x_3]$ are co-ordinates of two copies of $\Pn^1$). It's not hard to see the graph of $\varphi$ is given by the inverse image of $\D$ under $\id \times \varphi$. $$\Gamma(\varphi)= (\id \times \varphi)^{- 1}(\D)$$ Since the graph is closed it's inverse image will also be closed. Thus the graph is a closed set and under zariski topology any closed set is given by vanishing of some set of polynomials. This will help us to identify $\Gamma(\varphi)$ as a subvariety of $\Pn^1\times \Pn^1$. If $\varphi$ is given by $[x:y]\to [f(x,y):g(x,y)]$ then the graph can be given by the image of following vanishing set under segre embedding $$\qty{[x_0:x_1:x_2:x_3]: x_2 =f(x_0,x_1), x_3=g(x_0,x_1)}$$ 

\vspace*{0.2cm}

\noindent If, $\varphi$ given by $[x,y]\mapsto [x^2:y^2]$ the image of $([x:y],[x^2,y^2])$ is $[x^3:xy^2:yx^2:y^3]$(image under segre embedding). Which is rational curve $\Pn^1 \to \Pn^3$, a sub-variety of $\Pn^3$.
$$\Gamma(\varphi)\simeq \text{Rational curve in }\Pn^3$$

\begin{prob} %trishan 3
    (UAG 5.13) Study the embedding $\varphi : \Pn^2 \to \Pn^5$ given by $[x:y:z] \mapsto [x^2:xy:xz:y^2:yz:z^2]$ and prove that $\varphi$ is an isomorphism. Prove that the lines of $\Pn^2$ go over the conics of $\Pn^5$ and the conics go over the twisted quartics of $\Pn^5$.

    \vspace*{0.2cm}

    \noindent For any line $\ell \subset \Pn^2$, write $\pi(\ell) \subseteq \Pn^5$ for the projective plane spanned by the conics $\varphi(\ell)$. Prove that union of $\pi(\ell)$ taken over all $\ell \subset \Pn^2$ is a cubic hypersurface $\Sigma \subseteq \Pn^5$.
\end{prob}

\sol. Consider the following vanishing set on $\Pn^5$, $$S = V(t_0t_3-t_1^2, t_3t_5-t_4^2,t_0t_5-t_2^2,t_1t_2-t_0t_4,t_1t_4-t_3t_2,t_2t_4-t_5t_1)$$
It's not hard to see $\im \varphi \subset S$. Now note that the map $\varphi$ gives us a surjective map between the following vector spaces, \[
 \qty{\text{homogeneous quadratic polynomials in }t_0,\cdots,t_5} \to  \qty{\text{homogeneous quartics in } x,y,z}    
\] 
The first V.S is of dimension $21$ and the later one has dimension $15$. Thus the kernal has dimension $6$. Now note that the polynomials defining $S$ are linearly independent. So, $\im \varphi = S$. Thus the image of $\varphi$ is given by the variety $S$. Now take the map $\psi : S \to \Pn^3$ that maps $[t_0:\cdots:t_5] \to [t_0:t_1:t_2]$ works as the inverse map of $\varphi$ (it is defined except for $[0:0:0:0:0:1]$). So, $\varphi$ is an isomorphism. Any line in $\Pn^2$ can be given by the set $\qty{[x:y:ax+by]}$, image of this in $S$ is intersection of conics which will be again a conic (it can be degenerate). Any conic in $\Pn^2$ can be re-parametrized so that it is given by $[u^2:uv:v^2]$. It's image in $S$ is twisted quardics. 

\vspace*{0.2cm}

\noindent To do the last part we can also identify $S$ as the following set,\[S = \qty{ [t_0:t_1:\cdots:t_5] \in \Pn^5:\operatorname{rank }\begin{pmatrix}
    t_0 & t_1 & t_2 \\
    t_1 & t_3 & t_4 \\
    t_2 & t_4 & t_5
\end{pmatrix} \leq 1}\] From the above identification of $S$ we can say, $\cup_{\ell \subset \Pn^2}\pi(\ell)$ is given by $\det \begin{pmatrix}
    t_0 & t_1 & t_2 \\
    t_1 & t_3 & t_4 \\
    t_2 & t_4 & t_5
\end{pmatrix}=0$. This clearly determines a hyper-surface in $\Pn^5$. $\hfill \blacksquare$

\end{document}